\documentclass[a4paper,12pt]{article}
\usepackage{amsmath}
\usepackage{textcomp}
\usepackage{fancyhdr}
\usepackage{amssymb}
%\usepackage{psfig}
%\usepackage{amsmath}
%\usepackage[spanish]{babel} 
\usepackage{graphics}                 % Packages to allow inclusion of graphics
\RequirePackage{graphicx}
\usepackage{wrapfig}
\usepackage{caption}
\usepackage{comment}
\usepackage[spanish]{babel}
\spanishdecimal{.}
\usepackage{tikz}
\usepackage[american]{circuitikz}

\pagestyle{fancyplain} 
\long\def\symbolfootnote[#1]#2{\begingroup%
\def\thefootnote{\fnsymbol{footnote}}\footnote[#1]{#2}\endgroup} 

\hyphenation{ co-rres-pon-dien-tes cons-ta tem-pe-ra-tu-ra
              i-rre-ver-si-ble su-mi-nis-trar re-a-li-zan-do
              re-pre-sen-tar ne-ce-sa-rios pen-dien-tes
              res-pec-to di-fe-ren-tes de-sa-rro-lla-do
              io-ni-za-cion ra-san-tes co-li-sio-na-les
              de-ta-lla-re-mos par-ti-cu-lar es-cri-tu-ra
              De-sa-rro-lla-re-mos nues-tro nues-tros o-pe-ra-do-res
              de-ge-ne-ra-cio-nes de-ge-ne-ran a-su-mien-do 
              su-per-fi-cie co-rres-pon-dien-te po-si-ti-va
              a-pro-xi-ma-da-men-te }


\let\svthefootnote\thefootnote



\begin{document}

\lhead[\fancyplain{}]{Ejercicios}
\rhead[\fancyplain{}]{F5}

\begin{center}
{\Large \bf F\'{\i}sica 5$^{\mathrm{to}}$} \\
{\large \bf Ejercicios: Ley de Ohm y Circuitos CC}
\end{center}

%%%%%%%%%%%%%%%%%%%%%%%%%%%%%%%%%%%%%%%%%%%%%%%%%%%%%%%%%%%%%%%%%%%%%%%%
%%%%%%%%%%%%%%%%%%%%%%%%%%%%%%%%%%%%%%%%%%%%%%%%%%%%%%%%%%%%%%%%%%%%%%%%
\begin{enumerate}

\item 
\label{probcc}
\begin{enumerate}
\item Tom\'a la primer letra de tu nombre y asignale un n\'umero entero $a$.
Este n\'umero va a ser $a=1$ si tu nombre comienza con A, 2 si comienza con B, 
3 si es C, etc.
\item Asign\'a, de la misma forma, un n\'umero entero $b$ a tu apellido.
\item Calcul\'a $R=a+10\,b$.
\item Ese valor de $R$ es el que vas a utilizar en este ejercicio. \\
\end{enumerate}

\vspace{0.01\textheight}
Calcular los valores que miden el volt\'{\i}metro y 
el amper\'{\i}metro de la
Figura~\ref{fig:circuito}, para el caso en que la llave $LL1$ 
est\'a abierta, y para cuando est\'a cerrada.

\begin{circuitikz}[scale=1]
\draw
(0,2) to [battery1,l_=10 V, *-*] (0,-2)  
(0,2) to [R, l=R, *-*] (2,2) 
(0,-2) -- (2,-2)
(2,1) -- (2,3)
(2,1) to [R,l=R, *-*] (5,1) 
(5,1) -- (6,1)
(2,3) to [ammeter, *-*] (4,3) 
(4,3) to [R,l=2R, *-*] (6,3) 
(6,1) -- (6,3)
(6,2) to [R,l=$\frac{R}{2}$, *-*] (10,2) 
(7,1) to [voltmeter, *-*] (9,1) 
(7,1) to [short,*-*] (7,2)
(9,1) to [short,*-*] (9,2)
(10,-2) -- (10,2)
(0,-2) to [R, l=3R, *-*] (10,-2) 
;
%llave
\draw
(5,1) to [R, l_=3R, *-*] (6,-0.5)
(6,-0.5) to [normal open switch, l=LL1, *-*] (7,-2)
;
\end{circuitikz}
\captionof{figure}{Circuito del Problema ~\ref{probcc}}
\label{fig:circuito}

\vspace{1cm}
%%%%%%%%%%%%%%%%%%%%%%%%%%%%%%%%%%%%%%%%%%%%%%%%%%%%%%%%%%%%%%%%%%%%%%%%
\item Calcular la potencia disipada por la resistencia $2R$ en 
ambos escenarios (llave LL1 abierta y cerrada).


\vspace{1cm}
%%%%%%%%%%%%%%%%%%%%%%%%%%%%%%%%%%%%%%%%%%%%%%%%%%%%%%%%%%%%%%%%%%%%%%%%
\item (Puntaje Extra): 
Un calentador el\'ectrico cuya resistencia es $R= 20 \, \Omega$ 
se conecta a la red domiciliaria (220 V) para hervir una taza de 
agua. Calcular aproximadamente el tiempo que tarda y el 
costo de este proceso. 

\vspace{0.25 cm}
\underline{Ayuda}:\\
1 Calor\'{\i}a = 4.184 Joules \\
4 Tazas $\approx$ 1 litro

\end{enumerate}
%%%%%%%%%%%%%%%%%%%%%%%%%%%%%%%%%%%%%%%%%%%%%%%%%%%%%%%%%%%%%%%%%%%%%%%%
\end{document}
