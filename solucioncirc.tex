\documentclass[a4paper,12pt]{article}
\usepackage{amsmath}
\usepackage{textcomp}
\usepackage{fancyhdr}
\usepackage{amssymb}
%\usepackage{psfig}
%\usepackage{amsmath}
%\usepackage[spanish]{babel} 
\usepackage{graphics}                 % Packages to allow inclusion of graphics
\RequirePackage{graphicx}
\usepackage{wrapfig}
\usepackage{caption}
\usepackage{comment}
\usepackage[spanish]{babel}
\spanishdecimal{.}
\usepackage{tikz}
\usepackage[american]{circuitikz}

\pagestyle{fancyplain} 
\long\def\symbolfootnote[#1]#2{\begingroup%
\def\thefootnote{\fnsymbol{footnote}}\footnote[#1]{#2}\endgroup} 

\hyphenation{ co-rres-pon-dien-tes cons-ta tem-pe-ra-tu-ra
              i-rre-ver-si-ble su-mi-nis-trar re-a-li-zan-do
              re-pre-sen-tar ne-ce-sa-rios pen-dien-tes
              res-pec-to di-fe-ren-tes de-sa-rro-lla-do
              io-ni-za-cion ra-san-tes co-li-sio-na-les
              de-ta-lla-re-mos par-ti-cu-lar es-cri-tu-ra
              De-sa-rro-lla-re-mos nues-tro nues-tros o-pe-ra-do-res
              de-ge-ne-ra-cio-nes de-ge-ne-ran a-su-mien-do 
              su-per-fi-cie co-rres-pon-dien-te po-si-ti-va
              a-pro-xi-ma-da-men-te }


\let\svthefootnote\thefootnote



\begin{document}

\lhead[\fancyplain{}]{Examen 5}
\rhead[\fancyplain{}]{F5}

\begin{center}
{\Large \bf F\'{\i}sica 5$^{\mathrm{to}}$} \\
{\large \bf Soluci\'on Examen: Ley de Ohm y Circuitos CC}
\end{center}

%%%%%%%%%%%%%%%%%%%%%%%%%%%%%%%%%%%%%%%%%%%%%%%%%%%%%%%%%%%%%%%%%%%%%%%%
\begin{enumerate}

\item 

\begin{enumerate}
 \item Con LL1 Abierta: 

\label{probcca}
\begin{circuitikz}[scale=1]
\draw
(0,2) to [battery1,l_=10 V, *-*] (0,-2)  
(0,2) to [R,l=R, *-*] (2,2) 
(0,-2) -- (2,-2)
(2,1) -- (2,3)
(2,1) to [R,l=R, *-*] (5,1) 
(5,1) -- (6,1)
(2,3) to [ammeter, *-*] (4,3) 
(4,3) to [R,l=2R, *-*] (6,3) 
(6,1) -- (6,3)
(6,2) to [R,l=$\frac{R}{2}$, *-*] (10,2) 
(7,1) to [voltmeter, *-*] (9,1) 
(7,1) to [short,*-*] (7,2)
(9,1) to [short,*-*] (9,2)
(10,-2) -- (10,2)
(0,-2) to [R, l=3R, *-*] (10,-2) 
;
%llave
%\draw
%(5,1) to [R, l_=500 $\Omega$, *-*] (6,-0.5)
%(6,-0.5) to [normal open switch, l=LL1, *-*] (7,-2)
%;
\end{circuitikz}
\captionof{figure}{Circuito del Problema ~\ref{probcca}}
\label{fig:circuitoa}

\begin{eqnarray}
R_{par} &=& \frac{2R \times R}{2R + R}= \frac{2}{3}R \nonumber \\
R_{tot} &=& R + R_{par} + \frac{R}{2} + 3R = R \left(
1 + \frac{2}{3} + \frac{1}{2} + 3 \right) = 
R \left( \frac{6 + 4 + 3 + 18}{6} \right) = \frac{31}{6} R
\nonumber \\
I &=& \frac{V}{R_{tot}} = \frac{60 \, \mathrm{Volt}}{31 \, R \,\, (\Omega)} 
\nonumber 
\end{eqnarray}
La corriente en el amper\'{\i}metro es la corriente total, divida por 
3, ya que \'esta se bifurca en dos partes, $\frac{1}{3}$ circula por donde 
haya mayor resistencia, y $\frac{2}{3}$ por la rama con menor resistencia. 
Si denominamos $V_{R_{par}}$ a la diferencia 
de potencial el\'ectrico entre las resistencias en paralelo:
\begin{eqnarray}
V_{R_{par}} &=& I \times R_{par} \nonumber \\
\frac{I_{amp}}{I} &=& \frac{\frac{V_{R_{par}}}{2R}}{I} =  
\frac{\frac{I \times R_{par} }{2R}}{I} = \frac{R_{par}}{2R} = 
\frac{\frac{2}{3}R}{2R} = \frac{1}{3} \nonumber \\
I_{amp} &=& \frac{20 \, \mathrm{Volt}}{31 \, R  \,\, (\Omega)} \nonumber  \, .
\end{eqnarray}
El volt\'{\i}metro, por su parte, marca la ca\'{\i}da de tensi\'on 
sobre la resistencia:
\begin{eqnarray}
V_{volt} = I \cdot \frac{R}{2} = 
\frac{60 \, \mathrm{Volt}}{31 \, R } \times \frac{R}{2} = 
\frac{30}{31} \, \mathrm{Volt} = 0.97 \, \mathrm{Volt}
\nonumber \, .
\end{eqnarray}
Este valor no depende de $R$. >Puede ser correcto este resultado?

 \item Con LL1 Cerrada: \\
\label{probccc}
En este caso, el circuito se puede dibujar como:

\begin{circuitikz}[scale=1]
\draw
(0,2) to [battery1,l_=10 V, *-*] (0,-2)  
(0,2) to [R,l=R, *-*] (2,2) 
(0,-2) -- (2,-2)
(2,1) -- (2,3)
(2,1) to [R,l=R, *-*] (6,1) 
(2,3) to [ammeter, *-*] (4,3) 
(4,3) to [R,l=2R, *-*] (6,3) 
(6,1) -- (6,3)
(6,2) to [R,l=$\frac{R}{2}$, *-*] (10,2) 
(7,1) to [voltmeter, *-*] (9,1) 
(7,1) to [short,*-*] (7,2)
(9,1) to [short,*-*] (9,2)
(10,-2) -- (10,2)
(0,-2) to [R, l=3R, *-*] (10,-2) 
;
%sinllave
\draw
(7,0) to [R,l_=3R, *-*] (9,0)
(7,0) -- (7,1) 
(9,0) -- (9,1)
;
\end{circuitikz}
\captionof{figure}{Circuito del Problema ~\ref{probccc}}
\label{fig:circuitoc}

\begin{eqnarray}
R_{par_{2}} &=& \frac{ \frac{R}{2} \times 3R}{\frac{R}{2}+3R} = 
\frac{ \frac{3}{2} }{\frac{7}{2} }R = \frac{3}{7} R \nonumber \\
R_{tot} &=& R  + R_{par} + R_{par_{2}} + 3R = 
R \left(\, 1 + \frac{2}{3} + \frac{3}{7} + 3 \right) = 
\frac{107}{21} R \nonumber \\
I &=& \frac{V}{R_{tot}} = \frac{210 \, \mathrm{Volt}}{107 \, R \,\, (\Omega)}  
\nonumber 
\end{eqnarray}
La corriente en el amper\'{\i}metro es la corriente total, divida por 
3:
\begin{eqnarray}
I_{amp} = \frac{I}{3} = \frac{70 \, \mathrm{Volt}}{107 \, R \,\, (\Omega)}  
\nonumber  \, .
\end{eqnarray}
El volt\'{\i}metro, por su parte, marca la ca\'{\i}da de tensi\'on 
sobre la resistencia equivalente $R_{par_{2}}$:
\begin{eqnarray}
V_{volt} = I_{tot} \cdot R_{par_{2}} = 
\frac{210 \, \mathrm{Volt}}{107 \, R }  \cdot \frac{3}{7} R = 
\frac{90}{107} \,,\, \mathrm{Volt} = 0.84 \,\, \mathrm{Volt} 
\nonumber 
\end{eqnarray}


\end{enumerate}


\vspace{1cm}
%%%%%%%%%%%%%%%%%%%%%%%%%%%%%%%%%%%%%%%%%%%%%%%%%%%%%%%%%%%%%%%%%%%%%%%%
\item 
Si sabemos la corriente que circula por la resistencia, podemos 
calcular la potencia:
\begin{eqnarray}
W &=& I^2_{amp} \times 2 R 
\nonumber
\end{eqnarray}
\begin{enumerate}
\item Llave LL1 abierta:
\begin{eqnarray}
W &=& I^2_{amp} \times 2 R = 
\left( \frac{20 \, \mathrm{Volt}}{31 \, R } \right)^2 \times 2 R =  
\frac{800}{961 \, R} = \frac{0.83}{R} \,\, \mathrm{Watt} \nonumber  \, .
\end{eqnarray}

\item Llave LL2 cerrada: 
\begin{eqnarray}
W &=& I^2_{amp} \times 2 R = 
\left(  \frac{70 \, \mathrm{Volt}}{107 \, R}  \right)^2 \times 2 R =  
\frac{9800}{11449 \, R} = \frac{0.86}{R} \,\, \mathrm{Watt} \nonumber  \, .
\end{eqnarray}

\end{enumerate}
\vspace{1cm}
%%%%%%%%%%%%%%%%%%%%%%%%%%%%%%%%%%%%%%%%%%%%%%%%%%%%%%%%%%%%%%%%%%%%%%%%
\item La energ\'{\i}a total necesaria para calentar una taza de 
agua, desde 20 $^\circ{C}$ hasta 100 $^\circ{C}$ es
\begin{eqnarray}
Q = m \, c_v \, \Delta T = 250 \, g \,\,  4.184 \, 
\frac{J}{g \, ^\circ{C}}   
\, \, (100 - 20) \, ^\circ{C}  = 83.7 \, kJ \nonumber 
\end{eqnarray}
El KWh es una unidad de energ\'{\i}a, correspondiente a utilizar 
una potencia de 1kW durante 1 hora (3600 segundos):
\begin{eqnarray}
1 \, KWh = 1 \, \frac{kJ}{sec} \, 3600 \, sec = 3600 \, kJ 
\nonumber \, .
\end{eqnarray}
Por lo tanto, la energ\'{\i}a que se necesita para calentar la taza, 
en kWh es:
\begin{eqnarray}
Q \, [KWh] = \frac{Q \, [kJ] }{3600 \, kJ/KWh} = 0.023 \, KWh \, . \nonumber
\end{eqnarray}
Multiplicando por 3 pesos, nos da un costo de 0.07 $\$$.

Para obtener el tiempo que tarda el calentador, necesitamos calcular 
la potencia del mismo:
\begin{eqnarray}
P = \frac{V^2}{R} = \frac{220^2}{20} = 2420 \, W = 2.42 \, kW 
\nonumber \, .
\end{eqnarray}
El tiempo en segundos, es 
\begin{eqnarray}
t = \frac{Q}{P} = \frac{83.7 \, kJ}{2.42 \, kJ/s} = 35 \,\, \mathrm{sec}
\nonumber \, .
\end{eqnarray}

\end{enumerate}
%%%%%%%%%%%%%%%%%%%%%%%%%%%%%%%%%%%%%%%%%%%%%%%%%%%%%%%%%%%%%%%%%%%%%%%%
\end{document}
